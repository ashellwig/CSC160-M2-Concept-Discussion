\documentclass[12pt]{report}
  \usepackage[english]{babel}
  \usepackage[utf8]{inputenc}
  \usepackage[dvipsnames]{xcolor}
  \usepackage{hyperref}
  \usepackage{listings}
  \usepackage{parcolumns}
  \usepackage{algorithm}
  \usepackage{algorithmicx}
  \usepackage{algpseudocode}
  \usepackage{enumitem}
  \usepackage{geometry}
  \usepackage{soul}
  \usepackage{graphicx}
  \usepackage{enumitem}
  \usepackage{csquotes}
  \usepackage{bookmark}
  \usepackage{mdframed}
  \usepackage{mathtools}
  \usepackage{amsmath}
  \usepackage{amsthm}
  \usepackage[toc]{appendix}
  \usepackage[
    backend=biber,%
    style=apa%
  ]{biblatex}

  \colorlet{light-gray}{gray!10}

  % Bibliography Setup
  \addbibresource{main.bib}
  \newcommand{\CiteSection}[2]{%
    \parencite[~\S {#2}]{#1}
  }

  % Theorem Environments
  \theoremstyle{definition}
  \newtheorem*{defn*}{Definition}
  \theoremstyle{plain}
  \newtheorem*{equ*}{Equation}
  \theoremstyle{plain}
  \newtheorem*{examp*}{Example}

  % Definitions for Algorithmic Environments
  \algdef{SE}[VARIABLES]{GVariables}{EndGVariables}
    {\algorithmicvariables}
    {\algorithmicend\ \algorithmicvariables}
  \algnewcommand{\algorithmicvariables}{\textbf{global variables}}

  \algdef{SE}[VARIABLES]{LVariables}{EndLVariables}
    {\algorithmiclvariables}
    {\algorithmicend\ \algorithmiclvariables}
  \algnewcommand{\algorithmiclvariables}{\textit{local variables}}

  \renewcommand{\algorithmicrequire}{\textbf{Input:}}
  \renewcommand{\algorithmicensure}{\textbf{Output:}}
  \renewcommand\thealgorithm{}

  % Settings for math-mode
  \makeatletter
  \def\mathcolor#1#{\@mathcolor{#1}}
  \def\@mathcolor#1#2#3{%
    \protect\leavevmode
    \begingroup
      \color#1{#2}#3%
    \endgroup
  }
  \makeatother

  % Image Directory
  \graphicspath{ {screenshots/} }
  % Hyperlink Setup
  \hypersetup{
    colorlinks = true,
    urlcolor = blue,
    linkcolor = blue,
    citecolor = blue
  }
  % Syntax-Highlight for Code Snippets
  \lstset{
    breaklines=true,
    captionpos=b,
    frame=tb,
    tabsize=4,
    numbers=left,
    showstringspaces=false,
    commentstyle=\color{Red},
    keywordstyle=\color{Violet},
    stringstyle=\color{OliveGreen},
    backgroundcolor=\color{light-gray}
  }

  % Page and Text Layout
  \geometry{%
    a4paper,%
    top=1in,%
    bottom=1in,%
    left=1in,%
    right=1in%
  }
  \setlength{\headheight}{15pt}

  \newenvironment{ldefinitions}
    {\left.\begin{aligned}}
    {\end{aligned}\right\rbrace}

  \title{Module 2 Concept Discussion}
  \author{Ashton Hellwig}
  \date{\today}

\begin{document}
  \maketitle
  \tableofcontents
%   \lstlistoflistings
  \newpage


  \part{Initial Post}

    \section{Research Prompt}
      \begin{mdframed}[backgroundcolor=green!20]
        You can use the selection structures in this chapter interchangeably
          for many circumstances. There are a few occasions when one is clearly
          preferable to another based on input data. Some of these situations
          were touched on in the switch statement discussion. Review these, give
          an overview of all of the selection structures discussed in the
          chapter, and then discuss how to determine which structure to use when
          there is no clear indication. How much of the choice should be based
          on personal preference and how much on conventional practices? Are
          there performance issues as well?
      \end{mdframed}

    \section{Response}
      There are numerous selection structures available for use in modern C++,
        and for the most part they are interchangable. There are certain
        situations in which certain structures will perform better than others.

      \subsection{\texttt{if} Statements}
        If one wishes to select which statement to execute based on one simple
          condition, this is called a \hl{one-way} selection
          \CiteSection{malik_2015}{4-1d}. When attempting to select a statement
          based on a single condition (or a compound expression) while assigning
          a statement to execute if the conditions are not met, we use a
          \hl{two-way} selection structure, ``\texttt{if...else...}''
          \CiteSection{malik_2015}{4-1e}.

        % ONE-WAY SELECTION
        \begin{examp*}[One-Way Selection]
          Suppose a piece of software wanted to ensure users were authenticated
            in order to visit a certain route on the server. This could be
            checked by writing an ``\texttt{if}'' statement as provided in
            Listing \ref{if:onewayselection}.
        \end{examp*}

        \begin{lstlisting}[%
          language=c++,
          caption={Example: One Way Selection},
          label={if:onewayselection}
        ]
if (user.isAuthenticated == true) // Expression (Decision Maker)
  router.push('/auth');           // (Action) Statement
        \end{lstlisting}

        % TWO-WAY SELECTION
        \begin{examp*}[Two-Way Selection]
          Suppose a piece of software wanted to ensure users were authenticated
            in order to visit a certain route on the server, and \emph{if the
            authentication check fails}, send them to the ``home'' route. Well,
            This could be checked by writing an ``\texttt{if...else}'' statement
            as provided in Listing \ref{if:twowayselection}.
        \end{examp*}

        \begin{lstlisting}[%
          language=c++,
          caption={Example: Two Way Selection},
          label={if:twowayselection}
        ]
if (user.isAuthenticated == true) {
  router.push('/auth');
} else {
  router.push('/home')
}
        \end{lstlisting}




  \newpage
  \part{Responses}

    % \section{Response 1}
    %   \begin{mdframed}[backgroundcolor=green!20]
    %     Reply to \textbf{} (\textit{Post ID:})
    %   \end{mdframed}
    %   Placeholder


    % \section{Response 2}
    %   \begin{mdframed}[backgroundcolor=green!20]
    %     Reply to \textbf{} (\textit{Post ID:})
    %   \end{mdframed}
    %   Placeholder


  % Bibliography
  \newpage
  \nocite{malik_2015}
  \printbibliography[
    heading=bibintoc,
    title={Bibliography}
  ]
\end{document}

\documentclass[12pt,parskip=half]{scrreprt}
  \usepackage[USenglish]{babel}
  \usepackage[utf8]{inputenc}
  \usepackage{scrhack}
  \usepackage{hyperref}
  \usepackage[dvipsnames]{xcolor}
  \usepackage{listings}
  \usepackage{parcolumns}
  \usepackage{algorithm}
  \usepackage{algorithmicx}
  \usepackage{algpseudocode}
  \usepackage{enumitem}
  \usepackage{geometry}
  \usepackage{soul}
  \usepackage{graphicx}
  \usepackage{enumitem}
  \usepackage{csquotes}
  \usepackage{bookmark}
  \usepackage{mdframed}
  \usepackage{mathtools}
  \usepackage{amsmath}
  \usepackage{amsthm}
  \usepackage[toc]{appendix}
  \usepackage[
    backend=biber,%
    style=ieee%
  ]{biblatex}

  % Bibliography Setup
  \addbibresource{main.bib}
  \newcommand{\CiteSection}[2]{%
    (\autocite{#1}, ~\S {#1})
  }

  % Theorem Environments
  \theoremstyle{definition}
  \newtheorem*{defn*}{Definition}
  \theoremstyle{plain}
  \newtheorem*{equ*}{Equation}

  % Definitions for Algorithmic Environments
  \algdef{SE}[VARIABLES]{GVariables}{EndGVariables}
    {\algorithmicvariables}
    {\algorithmicend\ \algorithmicvariables}
  \algnewcommand{\algorithmicvariables}{\textbf{global variables}}

  \algdef{SE}[VARIABLES]{LVariables}{EndLVariables}
    {\algorithmiclvariables}
    {\algorithmicend\ \algorithmiclvariables}
  \algnewcommand{\algorithmiclvariables}{\textit{local variables}}

  \renewcommand{\algorithmicrequire}{\textbf{Input:}}
  \renewcommand{\algorithmicensure}{\textbf{Output:}}
  \renewcommand\thealgorithm{}

  % Settings for math-mode
  \makeatletter
  \def\mathcolor#1#{\@mathcolor{#1}}
  \def\@mathcolor#1#2#3{%
    \protect\leavevmode
    \begingroup
      \color#1{#2}#3%
    \endgroup
  }
  \makeatother


  % Image Directory
  \graphicspath{ {screenshots/} }
  % Hyperlink Setup
  \hypersetup{
    colorlinks = true,
    urlcolor = blue,
    linkcolor = blue,
    citecolor = blue
  }
  % Syntax-Highlight for Code Snippets
  \lstset{
    backgroundcolor=\color{white},
    breaklines=true,
    captionpos=b,
    frame=tb,
    tabsize=4,
    % numbers=left,
    showstringspaces=false,
    commentstyle=\color{Red},
    keywordstyle=\color{Aquamarine},
    stringstyle=\color{ForestGreen}
  }

  % Page and Text Layout
  \geometry{%
    a4paper,%
    top=1in,%
    bottom=1in,%
    left=1in,%
    right=1in%
  }
  \setlength{\headheight}{15pt}

  \newenvironment{ldefinitions}
    {\left.\begin{aligned}}
    {\end{aligned}\right\rbrace}

  \title{Module 2 Concept Discussion}
  \author{Ashton Hellwig}
  \date{\today}

\begin{document}
  \maketitle
  \tableofcontents
%   \lstlistoflistings
  \newpage


  \part{Initial Post}

    \section{Research Prompt}
      \begin{mdframed}[backgroundcolor=green!20]
        You can use the selection structures in this chapter interchangeably
          for many circumstances. There are a few occasions when one is clearly
          preferable to another based on input data. Some of these situations
          were touched on in the switch statement discussion. Review these, give
          an overview of all of the selection structures discussed in the
          chapter, and then discuss how to determine which structure to use when
          there is no clear indication. How much of the choice should be based
          on personal preference and how much on conventional practices? Are
          there performance issues as well?
      \end{mdframed}

    \section{Response}
      Placeholder.


  \newpage
  \part{Responses}

    % \section{Response 1}
    %   \begin{mdframed}[backgroundcolor=green!20]
    %     Reply to \textbf{} (\textit{Post ID:})
    %   \end{mdframed}
    %   Placeholder


    % \section{Response 2}
    %   \begin{mdframed}[backgroundcolor=green!20]
    %     Reply to \textbf{} (\textit{Post ID:})
    %   \end{mdframed}
    %   Placeholder


  % Bibliography
  \newpage
  \nocite{malik_2015}
  \printbibliography[
    heading=bibintoc,
    title={Bibliography}
  ]
\end{document}

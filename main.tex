\documentclass[12pt]{report}
  \usepackage[english]{babel}
  \usepackage[utf8]{inputenc}
  \usepackage[dvipsnames]{xcolor}
  \usepackage{hyperref}
  \usepackage{listings}
  \usepackage{parcolumns}
  \usepackage{algorithm}
  \usepackage{algorithmicx}
  \usepackage{algpseudocode}
  \usepackage{enumitem}
  \usepackage{geometry}
  \usepackage{soul}
  \usepackage{graphicx}
  \usepackage{enumitem}
  \usepackage{csquotes}
  \usepackage{bookmark}
  \usepackage{mdframed}
  \usepackage{mathtools}
  \usepackage{amsmath}
  \usepackage{amsthm}
  \usepackage[toc]{appendix}
  \usepackage[
    backend=biber,%
    style=apa%
  ]{biblatex}

  \colorlet{light-gray}{gray!10}

  % Bibliography Setup
  \addbibresource{main.bib}
  \newcommand{\CiteSection}[2]{%
    \parencite[~\S {#2}]{#1}
  }

  % Theorem Environments
  \theoremstyle{definition}
  \newtheorem*{defn*}{Definition}
  \theoremstyle{plain}
  \newtheorem*{equ*}{Equation}
  \theoremstyle{plain}
  \newtheorem*{examp*}{Example}

  % Definitions for Algorithmic Environments
  \algdef{SE}[VARIABLES]{GVariables}{EndGVariables}
    {\algorithmicvariables}
    {\algorithmicend\ \algorithmicvariables}
  \algnewcommand{\algorithmicvariables}{\textbf{global variables}}

  \algdef{SE}[VARIABLES]{LVariables}{EndLVariables}
    {\algorithmiclvariables}
    {\algorithmicend\ \algorithmiclvariables}
  \algnewcommand{\algorithmiclvariables}{\textit{local variables}}

  \renewcommand{\algorithmicrequire}{\textbf{Input:}}
  \renewcommand{\algorithmicensure}{\textbf{Output:}}
  \renewcommand\thealgorithm{}

  % Settings for math-mode
  \makeatletter
  \def\mathcolor#1#{\@mathcolor{#1}}
  \def\@mathcolor#1#2#3{%
    \protect\leavevmode
    \begingroup
      \color#1{#2}#3%
    \endgroup
  }
  \makeatother

  % Image Directory
  \graphicspath{ {screenshots/} }
  % Hyperlink Setup
  \hypersetup{
    colorlinks = true,
    urlcolor = blue,
    linkcolor = blue,
    citecolor = blue
  }

  % Syntax-Highlight for Code Snippets
  %% UTF-8 Support
  \lstset{literate=%
    {á}{{\'a}}1 {é}{{\'e}}1 {í}{{\'i}}1 {ó}{{\'o}}1 {ú}{{\'u}}1
    {Á}{{\'A}}1 {É}{{\'E}}1 {Í}{{\'I}}1 {Ó}{{\'O}}1 {Ú}{{\'U}}1
    {à}{{\`a}}1 {è}{{\`e}}1 {ì}{{\`i}}1 {ò}{{\`o}}1 {ù}{{\`u}}1
    {À}{{\`A}}1 {È}{{\'E}}1 {Ì}{{\`I}}1 {Ò}{{\`O}}1 {Ù}{{\`U}}1
    {ä}{{\"a}}1 {ë}{{\"e}}1 {ï}{{\"i}}1 {ö}{{\"o}}1 {ü}{{\"u}}1
    {Ä}{{\"A}}1 {Ë}{{\"E}}1 {Ï}{{\"I}}1 {Ö}{{\"O}}1 {Ü}{{\"U}}1
    {â}{{\^a}}1 {ê}{{\^e}}1 {î}{{\^i}}1 {ô}{{\^o}}1 {û}{{\^u}}1
    {Â}{{\^A}}1 {Ê}{{\^E}}1 {Î}{{\^I}}1 {Ô}{{\^O}}1 {Û}{{\^U}}1
    {Ã}{{\~A}}1 {ã}{{\~a}}1 {Õ}{{\~O}}1 {õ}{{\~o}}1
    {œ}{{\oe}}1 {Œ}{{\OE}}1 {æ}{{\ae}}1 {Æ}{{\AE}}1 {ß}{{\ss}}1
    {ű}{{\H{u}}}1 {Ű}{{\H{U}}}1 {ő}{{\H{o}}}1 {Ő}{{\H{O}}}1
    {ç}{{\c c}}1 {Ç}{{\c C}}1 {ø}{{\o}}1 {å}{{\r a}}1 {Å}{{\r A}}1
    {€}{{\euro}}1 {£}{{\pounds}}1 {«}{{\guillemotleft}}1
    {»}{{\guillemotright}}1 {ñ}{{\~n}}1 {Ñ}{{\~N}}1 {¿}{{?`}}1
  }
  %% Appearance
  \lstset{
    breaklines=true,
    captionpos=b,
    frame=tb,
    tabsize=4,
    numbers=left,
    showstringspaces=false,
    commentstyle=\color{Red},
    keywordstyle=\color{Violet},
    stringstyle=\color{OliveGreen},
    backgroundcolor=\color{light-gray}
  }

  % Page and Text Layout
  \geometry{%
    a4paper,%
    top=1in,%
    bottom=1in,%
    left=1in,%
    right=1in%
  }
  \setlength{\headheight}{15pt}

  \lstMakeShortInline[language=c++,columns=fixed]|

  \newenvironment{ldefinitions}
    {\left.\begin{aligned}}
    {\end{aligned}\right\rbrace}

  \title{Module 2 Concept Discussion}
  \author{Ashton Hellwig}
  \date{\today}

\begin{document}
  \maketitle
  \tableofcontents
%   \lstlistoflistings
  \newpage


  \chapter{Initial Post}

    \section{Research Prompt}
      \begin{mdframed}[backgroundcolor=green!20]
        You can use the selection structures in this chapter interchangeably
          for many circumstances. There are a few occasions when one is clearly
          preferable to another based on input data. Some of these situations
          were touched on in the switch statement discussion. Review these, give
          an overview of all of the selection structures discussed in the
          chapter, and then discuss how to determine which structure to use when
          there is no clear indication. How much of the choice should be based
          on personal preference and how much on conventional practices? Are
          there performance issues as well?
      \end{mdframed}

    \section{Response}
      There are numerous selection structures available for use in modern C++,
        and for the most part they are interchangable. There are certain
        situations in which certain structures will perform better than others.

      \subsection{\texttt{if} Statements}
        If one wishes to select which statement to execute based on one simple
          condition, this is called a \hl{one-way} selection
          \CiteSection{malik_2015}{4-1d}. When attempting to select a statement
          based on a single condition (or a compound expression) while assigning
          a statement to execute if the conditions are not met, we use a
          \hl{two-way} selection structure, ``\texttt{if...else...}''
          \CiteSection{malik_2015}{4-1e}. When we are required to incorporate
          multiple selections, we use a \textit{nested}
          ``\texttt{if...else...}'' structure \CiteSection{malik_2015}{4-1l}.

        % ONE-WAY SELECTION
        \begin{examp*}[One-Way Selection]
          Suppose a piece of software wanted to ensure users were authenticated
            in order to visit a certain route on the server. This could be
            checked by writing an ``\texttt{if}'' statement as provided in
            Listing \ref{if:onewayselection}.
        \end{examp*}

        \begin{lstlisting}[%
          language=c++,
          caption={Example: One Way Selection},
          label={if:onewayselection}
        ]
if (user.isAuthenticated == true) // Expression (Decision Maker)
  router.push('/auth');           // (Action) Statement
        \end{lstlisting}

        % TWO-WAY SELECTION
        \begin{examp*}[Two-Way Selection]
          Suppose a piece of software wanted to ensure users were authenticated
            in order to visit a certain route on the server, and \emph{if the
            authentication check fails}, send them to the ``home'' route. Well,
            This could be checked by writing an ``\texttt{if...else}'' statement
            as provided in Listing \ref{if:twowayselection}.
        \end{examp*}

        \begin{lstlisting}[%
          language=c++,
          caption={Example: Two Way Selection},
          label={if:twowayselection}
        ]
if (user.isAuthenticated == true) {
  router.push('/auth');
} else {
  router.push('/home')
}
        \end{lstlisting}

        % Multi-WAY SELECTION
        \begin{examp*}[Multi-Way Selection]
          Suppose a game wanted to dictate what strength level to give its users
            based on his or her level in the game. If their level is 100, the
            user gets $\infty$ assigned to their strength. If the user is
            between levels $50$ and $99$, their strength is set to $1000$. If
            the user is between level $25$ and $49$, their strength is set to
            $500$. For levels $<25$, we will assign $250$ as their strength
            value. To assign the correct strength value to the user, we can use
            the \textit{nested} \texttt{if...else...} statement as described in
            Listing \ref{if:multiwayselection}.
        \end{examp*}

        \begin{lstlisting}[%
          language=c++,%
          caption={Example: Multi-Way Selection},%
          label={if:multiwayselection},%
          mathescape=true%
        ]
// Note: I am aware that $\infty$ is not a valid value.
// This is just for illustration purposes.
if (user.level >= 100) {
  user.strength = $\infty$;
  else {
    if (50 <= user.level <= 99) {
      user.strength = 1000;
    } else {
      if (25 <= user.level <= 49) {
        user.strength = 500;
      } else {
        if (user.level <= 24) {
          user.strength = 250;
        }
      }
    }
  }
}
        \end{lstlisting}

        The deciding factor between using a nested \texttt{if...else...}
          statement versus a series of \texttt{if} statements lies in that when
          using the series of statements, more than one statement could execute
          if multiple conditions are true. Using the nested
          \texttt{if...else...} statement allows us to prevent that from
          happening by only executing the first statement which matches the
          given constraints.

      %% SWITCH STRUCTURES
      \newpage
      \subsection{\texttt{switch} Structures}
        Suppose we have \textbf{a lot} of alternatives for selection in our
          program. Nested \texttt{if...else...} structures tend to get a bit
          heavy when using a lot of selection options, which is where C++`s
          |switch| statement comes in. First the expression is evaluated and
          that is used to select amongst the alternatives given in the
          statements following the reserved word, ``|case|''. See the below
          example and Listing \ref{switch:syntax} for basic syntax.

        \begin{examp*}[Switch Syntax]
          Suppose you wanted to alert users as to which group they belong to,
            \texttt{admin}, \texttt{user}, or \texttt{guest}. To do this in C++,
            we would use the basic snippet of a switch structure as seen in
            Listing \ref{switch:syntax}.
        \end{examp*}

        \begin{lstlisting}[%
          language=c++,%
          caption={Example: Basic Switch Structure Syntax},%
          label={switch:syntax},%
          mathescape=true%
        ]
switch (user.group) {
  case "admin":
    cout << "You are in the admin group." << endl;
    break;
  case "user":
    cout << "You are in the user group." << endl;
    break;
  case "guest":
    cout << "You are in the guest group." << endl;
    break;
}
        \end{lstlisting}



  % Replies to Other`s Responses
  % ! TEX root=../main.tex

\newpage
\chapter{Responses}

  \section{Response 1}
    \begin{mdframed}[backgroundcolor=green!20]
      Reply to \textbf{} (\textit{Post ID:})
    \end{mdframed}
    Placeholder.


  \section{Response 2}
    \begin{mdframed}[backgroundcolor=green!20]
      Reply to \textbf{} (\textit{Post ID:})
    \end{mdframed}
    Placeholder.



  % Bibliography
  \newpage
  \printbibliography[
    heading=bibintoc,
    title={Bibliography}
  ]
\end{document}
